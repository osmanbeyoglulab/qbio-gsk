\documentclass[10pt,landscape]{article}
\usepackage{multicol}
\usepackage{calc}
\usepackage{ifthen}
\usepackage[landscape]{geometry}
\usepackage{hyperref}

% To make this come out properly in landscape mode, do one of the following
% 1.
%  pdflatex latexsheet.tex
%
% 2.
%  latex latexsheet.tex
%  dvips -P pdf  -t landscape latexsheet.dvi
%  ps2pdf latexsheet.ps


% If you're reading this, be prepared for confusion.  Making this was
% a learning experience for me, and it shows.  Much of the placement
% was hacked in; if you make it better, let me know...


% 2008-04
% Changed page margin code to use the geometry package. Also added code for
% conditional page margins, depending on paper size. Thanks to Uwe Ziegenhagen
% for the suggestions.

% 2006-08
% Made changes based on suggestions from Gene Cooperman. <gene at ccs.neu.edu>


% To Do:
% \listoffigures \listoftables
% \setcounter{secnumdepth}{0}


% This sets page margins to .5 inch if using letter paper, and to 1cm
% if using A4 paper. (This probably isn't strictly necessary.)
% If using another size paper, use default 1cm margins.
\ifthenelse{\lengthtest { \paperwidth = 11in}}
	{ \geometry{top=.5in,left=.5in,right=.5in,bottom=.5in} }
	{\ifthenelse{ \lengthtest{ \paperwidth = 297mm}}
		{\geometry{top=1cm,left=1cm,right=1cm,bottom=1cm} }
		{\geometry{top=1cm,left=1cm,right=1cm,bottom=1cm} }
	}

% Turn off header and footer
\pagestyle{empty}
 

% Redefine section commands to use less space
\makeatletter
\renewcommand{\section}{\@startsection{section}{1}{0mm}%
                                {-1ex plus -.5ex minus -.2ex}%
                                {0.5ex plus .2ex}%x
                                {\normalfont\large\bfseries}}
\renewcommand{\subsection}{\@startsection{subsection}{2}{0mm}%
                                {-1explus -.5ex minus -.2ex}%
                                {0.5ex plus .2ex}%
                                {\normalfont\normalsize\bfseries}}
\renewcommand{\subsubsection}{\@startsection{subsubsection}{3}{0mm}%
                                {-1ex plus -.5ex minus -.2ex}%
                                {1ex plus .2ex}%
                                {\normalfont\small\bfseries}}
\makeatother

% Define BibTeX command
\def\BibTeX{{\rm B\kern-.05em{\sc i\kern-.025em b}\kern-.08em
    T\kern-.1667em\lower.7ex\hbox{E}\kern-.125emX}}

% Don't print section numbers
\setcounter{secnumdepth}{0}


\setlength{\parindent}{0pt}
\setlength{\parskip}{0pt plus 0.5ex}


% -----------------------------------------------------------------------

\begin{document}

\raggedright
\footnotesize
\begin{multicols}{3}


% multicol parameters
% These lengths are set only within the two main columns
%\setlength{\columnseprule}{0.25pt}
\setlength{\premulticols}{1pt}
\setlength{\postmulticols}{1pt}
\setlength{\multicolsep}{1pt}
\setlength{\columnsep}{2pt}

\begin{center}
     \Large{\textbf{Unix Command Line\\
      Cheat Sheet}} \\
\end{center}

\section{Files and Directories}
\emph{Doing things:} \\
\begin{tabular}{@{}ll@{}}
\verb!cd!  & change directory \\
\verb!ls!  & list \\
\verb!pwd! & print working directory \\
\verb!mv!  & move \\
\verb!rm!  & remove \\
\verb!cp!  & copy \\
\end{tabular}
\vspace{5mm}\\
\emph{The lingo:} \\
\begin{tabular}{@{}ll@{}}
\verb!\!  & root directory \\
\verb!~!  & home directory \\
\verb!.!  & this directory \\
\verb!..! & one directory back \\
\verb!*!  & wild card \\
\end{tabular}
\vspace{5mm}\\
\emph{Tips 'n tricks:} \\
\begin{tabular}{@{}ll@{}}
\verb!TAB!  & tab completion \\
\verb!man __!  & see the manual for that command \\
\verb!^A!  & beginning of line \\
\verb!^E!  & end of line \\
\verb!^U! & delete whole line \\
\verb!PS1='$'!  & reduce prompt to just \$ \\
\verb!clear!  & clean workspace (keeps history)\\
\verb!^C! & ABORT/CANCEL \\
\end{tabular}


\section{A bit more advanced}
\begin{tabular}{@{}ll@{}}
\verb!head!  & the beginning of a file \\
\verb!tail!  & the end of a file \\
\verb!echo!  & print on screen \\
\verb!wc!    & word count \\
\verb!sort!  & sort \\
\verb!history!  & print history \\
\end{tabular}


\newlength{\MyLen}
\settowidth{\MyLen}{\texttt{letterpaper}/\texttt{a4paper} \ }

\section{Pipes and filters}
\begin{tabular}{@{}ll@{}}
\verb!>!  & redirect \\
\verb!|!  & pipe \\
\end{tabular}
\vspace{2.5mm}\\
\emph{Examples:} \\
sort -n lengths.txt $>$ sorted-lengths.txt\\
wc -l *.txt $|$ sort -n $|$ head -5

\section{Looking at files in command line}
\begin{tabular}{@{}ll@{}}
\verb!cat!  & concatenates contents line by line, printing \\
\verb!more!  & shows file and takes back to command line \\
\verb!less!  &  shows file in whole window\\
\end{tabular}

\section{NANO - simplest text editor}
\begin{tabular}{@{}ll@{}}
\verb!^o!  & write out \\
\verb!^x!  & exit \\
\verb!^k!  & cut \\
\verb!^w!  & find \\
\end{tabular}

\section{Searching}
\begin{tabular}{@{}ll@{}}
\verb!grep!  & grep \\
\verb!find!  & find \\
\end{tabular}
\vspace{2.5mm}\\
\emph{Examples:} \\
grep -n -w "the" haiku.txt \\
find . -name '*.txt' \\
grep "FE" \$(find .. -name '*.pdb')

\section{Loops}
for datafile in *[AB].txt; do echo \$datafile stats-\$datafile; done

\section{Variations on these themes}

\subsection{list}
\begin{tabular}{@{}ll@{}}
\verb!ls -a!  & list even files starting with . \\
\verb!ls -l!  & list in long format (includes dates)\\
\verb!ls -F!  & list with slash after directories \\
\end{tabular}
\vspace{2.5mm}\\
\subsection{wc}
\begin{tabular}{@{}ll@{}}
\verb!wc -l!  & count lines in file \\
\verb!wc -w!  & count words in file\\
\verb!wc -c!  & count characters in file \\
\end{tabular}

\section{Shell scripting}
\begin{tabular}{@{}ll@{}}
\verb!#!  & starts comment lines \\
\verb!$1 or $2 etc.!  & indicates arguments to use\\
\verb!$@!  & indicates to use all arguments \\
\end{tabular}

\section{BONUS}
\begin{tabular}{@{}ll@{}}
\verb!cut!  & cuts \\
\verb!sed!  & seds\\
\verb!awk!  & awks \\
\end{tabular}

\rule{0.3\linewidth}{0.25pt}
\scriptsize

This is a command line cheat sheet made February 2016 for Gerstner Sloan Kettering Graduate Students. It is still a work in progress. And remember Googling is your friend!
%THIS FORMAT IS BORROWED FROM LATEX CHEATSHEET WITH CP INFO BELOW
%Copyright \copyright\ 2014 Winston Chang

%\href{http://www.stdout.org/~winston/latex/}{http://www.stdout.org/$\sim$winston/latex/}


\end{multicols}
\end{document}
